\documentclass[12pt, a4paper]{article}
\usepackage[english, russian]{babel}            % russian text
\usepackage{geometry}                           % page sizes
\usepackage{wrapfig}                            % wrap a figure with text
\usepackage{graphicx}                           % include graphics
\usepackage{subfiles}                           % subfiles tex
\usepackage{ragged2e}                           % ragged text
\usepackage{indentfirst}                        % indent in the first sentence of a paragraph
\usepackage{tikz}                               % \Let declaration
\usepackage{amssymb}                            % sets of real and complex numbers
\usepackage{amstext}                            % \Let declaration 
\usepackage[bookmarks]{hyperref}                % refs in table of contents
\usepackage{tocloft}                            % table of contents customization
\usepackage{amsthm}                             % theorems
\usepackage{amsmath}                            % cases
\usepackage{stackengine,scalerel}               % leq in circle
\usepackage{stackengine}                        % = with up and down notes
\input{insbox}                                  % image in proof


\graphicspath{ {./img/} }

\renewcommand\cftsecfont{\normalfont}
\renewcommand\cftsecpagefont{\normalfont}
\renewcommand{\cftsecleader}{\cftdotfill{\cftsecdotsep}}
\renewcommand\cftsecdotsep{\cftdot}
\renewcommand\cftsubsecdotsep{\cftdot}
\renewcommand\cfttoctitlefont{\hfill\Large\bfseries}
\renewcommand\cftaftertoctitle{\hfill\mbox{}}

\newlength{\tempheight}
\newcommand{\Let}[0]{%
\mathbin{\text{\settoheight{\tempheight}{\mathstrut}\raisebox{0.5\pgflinewidth}{%
\tikz[baseline,line cap=round,line join=round] \draw (0,0) --++ (0.4em,0) --++ (0,1.5ex) --++ (-0.4em,0);%
}}}}

\newcommand\stackequal[2]{%
  \mathrel{\stackunder[2pt]{\stackon[4pt]{=}{$\scriptscriptstyle#1$}}{%
  $\scriptscriptstyle#2$}}}

\newlength{\wdth}
\newcommand{\strike}[1]{\settowidth{\wdth}{#1}\rlap{\rule[.5ex]{\wdth}{.4pt}}#1}

\newcommand{\R}[0]{\mathbb{R}}
\renewcommand{\C}[0]{\mathbb{C}}
\newcommand{\V}[2]{\underset{#1}{\overset{#2}{V}}}
\DeclareMathOperator{\vp}{v.p.}
\DeclareMathOperator{\si}{si}
\DeclareMathOperator{\ci}{ci}
\DeclareMathOperator{\lcm}{lcm}

\def\dclesize{\ThisStyle{\raisebox{-.7pt}{\scalebox{1.45}{$\SavedStyle\bigcirc$}}}}
\def\dcle{\ensurestackMath{\stackon[0pt]{\leq}{\dclesize}}}
\def\oleq{\def\stacktype{L}\mathbin{\scalerel*{\dcle}{\dclesize}}}
\def\dce{\ensurestackMath{\stackon[0pt]{=}{\dclesize}}}
\def\oeq{\def\stacktype{L}\mathbin{\scalerel*{\dce}{\dclesize}}}

\theoremstyle{definition}
\newtheorem*{example}{Пример}
\newtheorem*{examples}{Примеры}
\newtheorem*{remark}{Замечание}

\theoremstyle{plain}
\newtheorem{thm}{Теорема}[section]
\newtheorem*{crl}{Следствие}

\numberwithin{equation}{section}

\newenvironment*{prop}[1]{{\parindent0pt \textbf{#1}\vspace{6pt}\par}}{}
\newenvironment*{note}{{\parindent0pt \textbf{Замечание}\vspace{6pt}\par}}{}

\newcommand\paperline{\par\noindent\makebox[\linewidth]{\rule{\paperwidth}{0.4pt}}\par}

\geometry{left=2cm}
\geometry{right=1.5cm}
\geometry{top=2cm}
\geometry{bottom=3cm}

\begin{document}
\subfile{tex/title.tex}
\tableofcontents
\subfile{tex/question1.tex}
\subfile{tex/question2.tex}
\subfile{tex/question3.tex}
\subfile{tex/question4.tex}
\subfile{tex/question5.tex}
\subfile{tex/question6.tex}
\subfile{tex/question7.tex}
\subfile{tex/question8.tex}
\subfile{tex/question9.tex}
\subfile{tex/question10.tex}
\subfile{tex/question11.tex}
\subfile{tex/question12.tex}
\subfile{tex/question13.tex}
\subfile{tex/question14.tex}
\subfile{tex/question15.tex}
\subfile{tex/question16.tex}
\subfile{tex/question17.tex}
\subfile{tex/question18.tex}
\subfile{tex/question19.tex}
\subfile{tex/question20.tex}
\subfile{tex/question21.tex}
\subfile{tex/question22.tex}
\subfile{tex/question23.tex}
\subfile{tex/question24.tex}
\subfile{tex/question25.tex}
\subfile{tex/question26.tex}
\subfile{tex/question27.tex}
\subfile{tex/question28.tex}
\subfile{tex/question29.tex}
\subfile{tex/question30.tex}
\subfile{tex/question31.tex}
\subfile{tex/question32.tex}
\subfile{tex/question33.tex}
\subfile{tex/question34.tex}
\subfile{tex/question35.tex}
\subfile{tex/question36.tex}
\subfile{tex/question37.tex}
\subfile{tex/question38.tex}
\subfile{tex/question39.tex}
\subfile{tex/question40.tex}
\subfile{tex/question41.tex}
\subfile{tex/question42.tex}
\subfile{tex/question43.tex}
\subfile{tex/question44.tex}
\subfile{tex/question45.tex}
\subfile{tex/question46.tex}

\end{document}
