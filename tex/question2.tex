\documentclass[../main.tex]{subfiles}
\begin{document}
\newpage
\section{Доказательство неравенств с помощью выпуклости (примеры); Неравенство Юнга.}
\begin{thm}[\hypertarget{thm:Yung}{Неравенство Юнга}]
    
    ~

    \[ \Let \; p, q>1,\quad \dfrac{ 1}{ p}+ \dfrac{ 1}{ q}=1\]
    Тогда
    \[ \forall \; x,y \geq 0\quad xy \leq \dfrac{ x^p}{ p}+ \dfrac{ y^q}{ q}\]
\end{thm}
\begin{proof}
    Если \( xy=0\), то неравенство очевидно верно. \( \Let \; xy >0\)
    Рассмотрим \( f\left( x\right)=\ln\left( x\right)\). Эта функция строго вогнутая, поэтому из определения вогнутости:
    \[ \ln \left( \dfrac{ x^p}{ p}+ \dfrac{ y^q}{ q}\right) \geq \dfrac{ 1}{ p} \ln \left( x^p\right)+ \dfrac{ 1}{ q} \ln \left( y^q\right)= \ln x+ \ln y = \ln \left( xy\right)\]

    Так как логарифм монотонно возрастает, это равносильно 
    \[ \dfrac{ x^p}{ p}+ \dfrac{ y^q}{ q} \geq xy\]
\end{proof}

Числа \( p\) и \( q\) из условия неравенства Юнга иногда называют \emph{сопряжёнными} или \emph{сопряжёнными показателями}. 

Возможно, это доказательство сойдёт за пример доказательства неравенств через выпуклость. Другого на лекции всё равно не было, так что имеем что имеем. 
\end{document}