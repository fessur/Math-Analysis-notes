\documentclass[../main.tex]{subfiles}
\begin{document}
\newpage
\section{Связь сходимости ряда с поведением его остатков. Линейность суммирования.}

\( \sum\limits_{ k=n}^{ \infty } a_k\) называется \emph{остатком числового ряда} \( \sum\limits_{ k=1}^{ \infty } a_k\).

\begin{thm}[Связь сходимости ряда с поведением его остатков]
    
    ~

    Следующие утверждения равносильны:
    \begin{enumerate}
        \item \( \sum\limits_{ k=1}^{ \infty } a_k\) сходится
        \item \( \exists \; n \in \N:\quad \sum\limits_{ k=n}^{ \infty } a_k\) сходится
        \item \(
            \begin{cases}
                \forall \; n \in \N\quad \sum\limits_{ k=n}^{ \infty } a_k \text{ сходится} \\
                \sum\limits_{ k=n}^{ \infty } a_k \underset{n \rightarrow \infty }{ \longrightarrow } 0
            \end{cases}
        \)
    \end{enumerate}
\end{thm}
\begin{proof}
    
    ~\\
    \( \boxed{ 3\Longrightarrow 2}:\) 
    
    Очевидно. 
    ~\\
    \( \boxed{ 2 \Longrightarrow 1}:\)

    \( \Let \; \tilde{ S}_N = \sum\limits_{ k=n}^{ N} a_k\). Тогда 
    \( S_N = \sum\limits_{ k=1}^{ N-1} a_k + \underbrace{\tilde{ S}_N}_{\text{сходится}} \Longrightarrow S_n \text{ сходится.}\)
    ~\\
    \( \boxed{ 1 \Longrightarrow 3}:\)

    \( \forall \; n \in \N\quad \tilde{ S}_N= \sum\limits_{ k=n}^{ N} a_k\). Тогда 
    \( \underbrace{\tilde{ S}_N}_{\text{сходится}}= S_N- \sum\limits_{ k=1}^{ n-1} a_k \Longrightarrow S_N \text{ сходится.}\)

    Доказали, что любой остаток сходится. Кроме того, если сделать предельный переход при \( N \longrightarrow \infty \):
    \[ \sum\limits_{ k=n}^{ \infty } a_k= \sum\limits_{ k=1}^{ \infty } a_k - \sum\limits_{ k=1}^{ n-1} a_k\]

    Если \( n \longrightarrow \infty \), то \( \sum\limits_{ k=1}^{ \infty } a_k \longrightarrow \sum\limits_{ k=1}^{ \infty } a_k\), т.к. не зависит от n, и \( \sum\limits_{ k=1}^{ n-1} a_k \longrightarrow \sum\limits_{ k=1}^{ \infty } a_k\) по определению. Поэтому \( \sum\limits_{ k=n}^{ \infty } a_k \underset{n \rightarrow \infty }{ \longrightarrow }0\).
\end{proof}

\begin{thm}[Линейность суммирования]
    
    ~

    \( \sum\limits_{ k=1}^{ \infty } a_k\), \( \sum\limits_{ k=1}^{ \infty } b_k\) - сходящиеся числовые ряды, \( \alpha , \beta \in \C\).

    Тогда
    \[ \sum\limits_{ k=1}^{ \infty } \left( \alpha a_k+ \beta b_k\right) \text{ сходится и } \sum\limits_{ k=1}^{ \infty } \left( \alpha a_k+ \beta b_k\right)= \alpha  \sum\limits_{ k=1}^{ \infty }a_k + \beta \sum\limits_{ k=1}^{ \infty } b_k\]
\end{thm}
\begin{proof}
    
    ~

    Ряд - это предел, а для пределов данное свойство выполнено.
\end{proof}

\begin{note}
    Сходящийся + Сходящийся = Сходящийся

    Сходящийся + Расходящийся = Расходящийся 
    
    Расходящийся + Расходящийся = ? \( \longrightarrow \) это зависит от конкретных рядов
\end{note}

\end{document}
