\documentclass[../main.tex]{subfiles}
\begin{document}
\newpage
\section{Интегрирование по частям для неопределенных интегралов; примеры.}
\begin{thm}[\hypertarget{thm:undef_by_parts}{Интегрирование по частям в неопределённом интеграле}]
    
    ~

    \( \Let \; u, v \in C^1\left( I\right),\quad I\text{ - невырожденный промежуток.}\)

    Тогда

    \[ \displaystyle\int\limits_{ }^{ } u\left( x\right)v'\left( x\right)dx=u\left( x\right)v\left( x\right)- \displaystyle\int\limits_{ }^{ } v\left( x\right)u'\left( x\right)dx\]

    Или по простому

    \[ \displaystyle\int\limits_{ }^{ } udv=uv- \displaystyle\int\limits_{ }^{ } vdu\]
\end{thm}
\begin{proof}
    
    ~

    Из условия следует, что \( uv'\) и \( vu'\) непрерывны, а значит имеют первообразные (это будет доказано позднее). 

    \[ \displaystyle\int\limits_{ }^{ } d\left( uv\right)= \displaystyle\int\limits_{ }^{ } \left( udv+vdu\right)= \displaystyle\int\limits_{ }^{ } udv + \displaystyle\int\limits_{ }^{ } vdu \implies \displaystyle\int\limits_{ }^{ } udv= \displaystyle\int\limits_{ }^{ } d\left( uv\right)- \displaystyle\int\limits_{ }^{ } vdu\]

    Но
    \[ \displaystyle\int\limits_{ }^{ } d \varphi = \varphi + C \implies \displaystyle\int\limits_{ }^{ } d\left( uv\right)=uv+ C \]

    Поэтому
    \[ \displaystyle\int\limits_{ }^{ } udv= uv + C - \displaystyle\int\limits_{ }^{ } vdu=uv- \displaystyle\int\limits_{ }^{ } vdu\]
    (константа внутри интеграла)
\end{proof}

\begin{example}
    \[ \displaystyle\int\limits_{ }^{ } \ln x\;dx=x \ln x- \displaystyle\int\limits_{ }^{ } xd\left( \ln x\right)=x \ln x- \displaystyle\int\limits_{ }^{ } dx=x \ln x -x+C\]
\end{example}

\begin{example}
    \begin{equation*}
        \begin{aligned}
            \displaystyle\int\limits_{ }^{ } \arctg x\;dx&=x \arctg x - \displaystyle\int\limits_{ }^{ } xd\left( \arctg x\right)=x\arctg x - \displaystyle\int\limits_{ }^{ } \dfrac{ x}{ x^2+1} dx=\\
            &=x\arctg x - \dfrac{ 1}{ 2} \displaystyle\int\limits_{ }^{ } \dfrac{ d\left(x^2+1\right)}{ x^2+1}=x\arctg x - \dfrac{ 1}{ 2} \ln \left( x^2+1\right)+C
        \end{aligned}
    \end{equation*}

\end{example}
\end{document}