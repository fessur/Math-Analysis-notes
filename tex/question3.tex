\documentclass[../main.tex]{subfiles}
\begin{document}
\newpage
\section{Неравенства Минковского для сумм.}
Пусть \( x\) - вектор из \( \R ^n\) (то есть \( x = \left( x_1, \ldots , x_n\right),\quad \forall \; i\quad x_i \in \R \)), \( p \geq 1\).
\emph{p-нормой вектора \(x\)} называют \( \left| \left| x\right|\right|_p=\left( \sum\limits_{ k=1}^{ n} \left| x_k\right|^p\right)^ \frac{ 1}{ p}\). 
Чаще всего используется \( p=2\), тогда норма - это длина вектора, в такой ситуации в обозначении нормы букву \( p\) опускают. 

\begin{thm}[\hypertarget{thm:minkovsky}{Неравенство Минковского (неравенство треугольника)}]

    ~
    
    \( \Let \; p \geq 1,\quad x, y \in \R ^n.\)

    Тогда 
    \[ \left(\sum\limits_{ k=1}^{ n} \left| x_k+y_k\right|^p\right)^ \frac{ 1}{ p} \leq \left( \sum\limits_{ k=1}^{ n} \left| x_k\right|^p\right)^ \frac{ 1}{ p} + \left( \sum\limits_{ k=1}^{ n} \left| y_k^p\right|\right)^ \frac{ 1}{ p} \]
    Или (то же самое, но другими словами)
    \[ \left| \left| x+y\right|\right|_p \leq \left| \left| x\right|\right|_p+\left| \left| y\right|\right|_p\]
\end{thm}
\begin{proof}

    ~

    Если \( p=1\), то неравенство превращается в "модуль суммы не превосходит суммы модулей". 

    Если \( p > 1\), то рассмотрим сопряжённое ему \( q:\quad \dfrac{ 1}{ p} + \dfrac{ 1}{ q} =1\). Заметим, что
    \[ q = \dfrac{ 1}{ 1- \frac{ 1}{ p} }= \dfrac{ p}{ p-1} \implies \left( p-1\right)q=p \]
    а также
    \[ 1- \dfrac{ 1}{ q} = \dfrac{ 1}{ p}  \implies p- \dfrac{ p}{ q} =1 \]

    Сначала воспользуемся тем, что модуль суммы не превосходит суммы модулей:
    \[ \left| \left| x+y\right|\right|_p^p= \sum\limits_{ k=1}^{ n} \left| x_k+y_k\right|^p= \sum\limits_{ k=1}^{ n} \left| x_k+y_k\right|^p\left| x_k+y_k\right| \leq \sum\limits_{ k=1}^{ n} \left| x_k+y_k\right|^{p-1}\left| x_k\right|+ \sum\limits_{ k=1}^{ n} \left| x_k+y_k\right|^{p-1}\left| y_k\right| \oleq\]

    По \hyperlink{thm:Gelder}{неравенству Гёльдера}:
    \[ \sum\limits_{ k=1}^{ n} \left| x_k+y_k\right|^{p-1}\left| x_k\right| \leq \left( \sum\limits_{ k=1}^{ n} \left| x_k+y_k\right|^{(p-1)q}\right)^ \frac{ 1}{ q} \left( \sum\limits_{ k=1}^{ n} \left| x_k\right|^p\right)^ \frac{ 1}{ p} \]
    \[ \sum\limits_{ k=1}^{ n} \left| x_k+y_k\right|^{p-1}\left| y_k\right| \leq \left( \sum\limits_{ k=1}^{ n} \left| x_k+y_k\right|^{(p-1)q}\right)^ \frac{ 1}{ q} \left( \sum\limits_{ k=1}^{ n} \left| y_k\right|^p\right)^ \frac{ 1}{ p} \]

    Поэтому, продолжая цепочку рассуждений:
    \begin{equation*}
        \begin{aligned}
            &\oleq \left( \sum\limits_{ k=1}^{ n} \left| x_k+y_k\right|^{(p-1)q}\right)^ \frac{ 1}{ q} \left( \sum\limits_{ k=1}^{ n} \left| x_k\right|^p\right)^ \frac{ 1}{ p} + \left( \sum\limits_{ k=1}^{ n} \left| x_k+y_k\right|^{(p-1)q}\right)^ \frac{ 1}{ q} \left( \sum\limits_{ k=1}^{ n} \left| y_k\right|^p\right)^ \frac{ 1}{ p} = \\
            &=\left(\left( \sum\limits_{ k=1}^{ n} \left| x_k+y_k\right|^p\right)^ \frac{ 1}{ p}\right)^ \frac{ p}{ q} \left( \sum\limits_{ k=1}^{ n} \left| x_k\right|^p\right)^ \frac{ 1}{ p} + \left(\left( \sum\limits_{ k=1}^{ n} \left| x_k+y_k\right|^p\right)^ \frac{ 1}{ p}\right)^ \frac{ p}{ q} \left( \sum\limits_{ k=1}^{ n} \left| y_k\right|^p\right)^ \frac{ 1}{ p} =\\
            &=\left| \left| x+y\right|\right|_p^ \frac{ p}{ q} \cdot \left| \left| x\right|\right|_p+ \left| \left| x+y\right|\right|_p^ \frac{ p}{ q} \cdot \left| \left| y\right|\right|_p
        \end{aligned}
    \end{equation*}

    Итого мы получили
    \[ \left| \left| x+y\right|\right|_p^p \leq \left| \left| x+y\right|\right|_p^ \frac{ p}{ q} \left( \left| \left| x\right|\right|_p+\left| \left| y\right|\right|_p\right)\]

    Вообще если \( \left| \left| x+y\right|\right|_p=0\), то неравенство Минковского очевидно (справа просто \( \geq 0\), слева 0). Если \( \left| \left| x+y\right|\right|_p \neq 0\):
    \[ \left| \left| x+y\right|\right|_p^{p- \frac{ p}{ q} } \leq \left| \left| x\right|\right|_p + \left| \left| y\right|\right|_p\]
    \[ \left| \left| x+y\right|\right|_p \leq \left| \left| x\right|\right|_p + \left| \left| y\right|\right|_p\]
\end{proof}
\end{document}