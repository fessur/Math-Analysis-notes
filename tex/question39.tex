\documentclass[../main.tex]{subfiles}
\begin{document}
\newpage
\section{Аддитивная функция промежутка и ее плотность. Критерий непрерывной плотности.}
\( \Let \left[ A,B\right] \subseteq  \R \qquad\Phi:\left\{ \left[ a,b\right]\right\}_{\left[ a,b\right]\in\left[ A,B\right]} \longrightarrow  \R\).

\(\ \Phi\) называется \emph{аддитивной функцией промежутка}, если 
\[ \forall \; a \leq c \leq b: \left[ a,b\right] \subseteq \left[ A,B\right]\quad\Phi \left( \left[ a,b\right]\right)= \Phi \left( \left[ a,c\right]\right)+ \Phi \left( \left[ c, b\right]\right)\]

\begin{examples}
    
    ~

    \begin{enumerate}
        \item \( \Phi \left( \left[ a,b\right]\right)=\left| b-a\right|\) - длина промежутка
        \item \( \Phi \left( \left[ a,b\right]  \right) = \begin{cases}
            1, \text{если } 0 \in \left[ a,b\right)\\ 
            0, \text{если } 0 \notin \left[ a,b\right) 
        \end{cases} \)
        \item \( f \in C\left[ A,B\right],\quad \Phi\left( \left[ a,b\right]\right)= \displaystyle\int\limits_{ a}^{ b} f(x)dx\) \\ Определённый интеграл - аддитивная функция промежутка
    \end{enumerate}
\end{examples}
\[ \Let \; \Phi: \left\{ \left[ a,b\right]\right\}_{ \left[ a,b\right] \subseteq \left[ A,B\right]} \longrightarrow \R,\quad \rho \in C\left[ A,B\right],\quad \forall \; \left[ a,b\right] \subseteq \left[ A,B\right]\quad \Phi \left( \left[ a,b\right]\right)= \displaystyle\int\limits_{ a}^{ b} \rho (x)dx\]
Тогда \( \rho\) называется \emph{плотностью} для \( \Phi \). 
\begin{remark}

    ~

    Не всякая аддитивная функция промежутка имеет плотность. 
    
    Функция из примера 2 плотности не имеет, потому что в интеграле из определения плотности можно зафиксировать нижний предел
    интегрирования. Зафиксируем его равным -10 и будем считать, что \( \Phi\) определена на всех подотрезках отрезка \( \left[ -10, 10\right]\). Тогда \( \forall \;x\in \left[ -10, 10\right]\quad \Phi (\left[ -10,x\right])= \displaystyle\int\limits_{ -10}^{ x} \rho (t)dt\) - 
    это верно из определения плотности. С другой стороны, \( \displaystyle\int\limits_{-10}^{ x} \rho (t)dt\) - это первообразная функции \( \rho\) в точке \( x\) на отрезке \( \left[ -10,10\right]\) по теореме Барроу. Значит первообразная функции \( \rho\) в точке \( x\) равна \( \Phi \left( \left[ -10,x\right]\right)\). Эта первообразная дифференцируема на \( \left[ -10,10\right]\)
    (потому что её производной по определению первообразной должна быть функция \( \rho\)), а значит эта первообразная должна быть непрерывной на \( \left[ -10, 10\right]\). А раз мы знаем, что первообразная функции \( \rho\) в точке \( x\) равна \( \Phi \left( \left[ -10, x\right]\right)\), давайте посмотрим, как ведёт себя \( \Phi\left( \left[ -10,x\right]\right) \).
    На полуинтервале \( \left( -10, 0\right]\) она равна 0, т.к. число 0 не содержится в полуинтервале \(\left[ -10, 0\right)\). На полуинтервале \( \left( 0, 10\right]\quad\Phi \left( \left[ -10, x\right]\right)\) равна 1, т.к. на соответствующих полуинтервалах 0 уже содержится. 
    Получается, что функция \( \Phi \left( [-10,x]\right)\) (которая равна первообразной \( \rho\) в точке \( x\)) имеет разрыв первого рода (скачок) в точке 0, что противоречит тому, что она должна быть непрерывной. 

    То есть чтобы аддитивная функция промежутка имела плотность, необходимо хотя бы чтобы функция \( F(x) = \displaystyle\int\limits_{ x_{ 0}}^{ x} \rho (t)dt\) была непрерывна на \( \left[ A,B\right]\). А какая аддитивная функция промежутка точно имеет плотность? Ответ на 
    этот вопрос даёт признак плотности.
\end{remark}

\begin{thm}[\hypertarget{thm:density}{Признак плотности}]
    
    \( \Let \; \Phi:\left\{ \left[ a,b\right]\right\}_{\left[ a,b\right] \subseteq \left[ A,B\right] } \longrightarrow \R\) - аддитивная функция промежутка,
    \( \rho(x) \in C\left[ A,B\right] \) и
    \[ \forall \;\left[ a, b\right] \subseteq \left[ A,B\right]\quad \min\limits_{ \left[ a,b\right]} \rho \cdot\left( b-a\right) \leq \Phi\left( \left[ a,b\right]\right) \leq \max\limits_{ \left[ a,b\right]} \rho\cdot\left( b-a\right) \]
    Тогда \( \rho(x)\) - плотность \( \Phi\).
\end{thm}
\begin{proof}
    
    ~

    Идея на самом деле в том, что если промежуток \( \left[ a,b\right]\) маленький, то \( \min\limits_{ \left[ a,b\right]} \rho \) и \( \max\limits_{ \left[ a,b\right]} \rho \) это почти одно и то же число, а раз 
    \( \Phi\left[ a,b\right]\) лежит между ними, то ему ничего не остаётся, кроме как равняться им. \( \rho\cdot\left( b-a\right)\) это площадь маленького прямоугольника при близких \( a, b\), а т.к. \( \Phi\) - аддитивная функция промежутка, 
    \( \Phi\left[ a,b\right]\) можно представить в виде суммы её на маленьких отрезках, где она равна площади маленького прямоугольника. А сумма площадей маленьких прямоугольников - это интеграл по Риману.

    \( \Let \; I= \displaystyle\int\limits_{ a}^{ b} \rho(x)dx\), \( X=\left\{ a=x_{ 0},x_{ 1}, \dots,x_{ n}=b\right\}\) - дробление \( \left[ a,b\right]\), \( T = \left\{ t_{ i}\right\}_{i=1}^{n}\) - его оснащение. Вспоминаем определение интеграла по Риману:
    \[ \forall \; \varepsilon >0\quad \exists \; \delta >0:\quad \lambda (X)< \delta \implies \left| I- \sum\limits_{ i=1}^{ n} \rho(t_i) \Delta x_i\right|< \varepsilon \implies I- \varepsilon <\sum\limits_{ i=1}^{ n} \rho\left( t_i\right) \Delta x_i< I+ \varepsilon \]

    Причём это верно для любого оснащения \( T\). Вспомним, что мы умеем оценивать \( \Phi\left( \left[ a,b\right]\right)\).
    \[ \Phi\left( \left[ a,b\right]\right)= \sum\limits_{ k=1}^{ n} \Phi\left( \left[ x_{k-1}, x_k\right]\right) \leq \sum\limits_{ k=1}^{ n} \max\limits_{ \left[ x_{k-1},x_k\right]}\rho\;\cdot \Delta x_k= \sum\limits_{ k=1}^{ n} \rho\left( c_k\right) \Delta x_k\]

    для некоторых \( c_k\in\left[ x_{k-1}, x_k\right]\). Заметим, что \( \sum\limits_{ k=1}^{ n} \rho\left( c_k\right) \Delta x_k\) - это сумма Римана, а интеграл \( I\) это как раз предел таких сумм, поэтому 
    \[ \sum\limits_{ k=1}^{ n} \rho\left( c_k\right) \Delta x_k < I + \varepsilon \]

    Рассуждая аналогично, \[ \Phi\left( \left[ a,b\right]\right)= \sum\limits_{ k=1}^{ n} \Phi\left( \left[ x_{k-1}, x_k\right]\right) \geq \sum\limits_{ k=1}^{ n} \min\limits_{ \left[ x_{k-1}, x_k\right]} \rho \;\cdot \Delta x_k= \sum\limits_{ k=1}^{ n} \rho\left( \tilde{c_k}\right) \Delta x_k>I- \varepsilon \]

    Таким образом, \[ \forall \; \varepsilon >0\quad I- \varepsilon <\Phi\left( \left[ a,b\right]\right)<I+ \varepsilon \implies \Phi\left( \left[ a,b\right]\right)=I \]
\end{proof}
\end{document}