\documentclass[../main.tex]{subfiles}
\begin{document}
\newpage
\section{Выпуклость и асимптота.}

В первом семестре была лемма о трёх хордах, поэтому сейчас доказывать её не будем, просто проговорим. 

\begin{thm}[\hypertarget{thm:three_hord}{Лемма о трёх хордах}]
    
    ~

    Если \( f\) - выпуклая (строго выпуклая) функция на промежутке \( I\), то \\
    \( \forall \; x_1, x_2, x \in I:\quad x_1<x<x_2\) верно
    \[ \dfrac{ f(x)-f(x_1)}{ x-x_1} \leq \dfrac{ f(x_2)-f(x_1)}{ x_2-x_1} \leq \dfrac{ f(x_2)-f(x)}{ x_2-x}\]

    В строгом случае все неравенства строгие. 
\end{thm}

\begin{crl}
        \hypertarget{thm:three_hord_crl}{~}
        Если \( f\) выпукла (строго выпукла) на \( I\), то \( F\left( x\right)= \dfrac{ f(x)-f(a)}{ x-a}\) возрастает (строго возрастает) на \( I \;\backslash \left\{ a\right\}\).
\end{crl}
\begin{proof}
    
    ~

    Посмотрим на строгий случай (нестрогий абсолюьно аналогично). Надо проверить, что \( \forall \; s,t \in I:\quad s<t\;\) верно \( \;F\left( s\right) < F\left( t\right)\). 

    Как могут быть расположены точки \( a,s,t\) относительно друг друга? Возможны 3 варианта:
    \begin{enumerate}
        \item \( s < t < a\)
        \item \( s < a < t\) 
        \item \( a < s < t\)
    \end{enumerate}

    Рассмотрим первый вариант, тогда сразу станет понятно, как рассматриваются остальные. Итак, 
    \( s < t < a\). Переобозначим \( x_1=s,\;x=t,\;x_2=a\). Мы хотим доказать, что \( F(x_1)<F(x)\), то есть что
    \[ \dfrac{ f(x_1)-f(x_2)}{ x_1-x_2}< \dfrac{ f(x)-f(x_2)}{ x-x_2}\]

    Домножим все числители и знаменатели на \( -1\):

    \[ \dfrac{ f(x_2)-f(x_1)}{ x_2-x_1}< \dfrac{ f(x_2)-f(x)}{ x_2-x}\]

    Это верно по второму неравенству в \hyperlink{thm:three_hord}{лемме о трёх хордах} (даже обозначения совпали!). 
\end{proof}

\begin{thm}[Выпуклость и ассимптота]
    
    ~

    \( \Let \; A \in \R\), \( f\) выпукла (строго выпукла) на \( \langle A, + \infty )\) и имеет ассимтоту \( a\left( x\right)=kx+b\) при \( x \rightarrow + \infty,\quad k,b \in \R \).
    
    Тогда
    \[ \forall \; x \in \langle A, + \infty )\quad f\left( x\right) \geq a\left( x\right)\quad (f\left( x\right)>a\left( x\right)\text{ в строгом случае})\]
\end{thm}
\begin{proof}
    Будем доказывать строгий случай, нестрогий доказывается аналогично (возможно даже чуть проще). 

    Рассмотрим функцию \( g(x)=f(x)-kx\). По определению ассимптоты 
    \[ \lim\limits_{ x \rightarrow + \infty }g(x)= \lim\limits_{ x \rightarrow + \infty } f(x)-kx =b\]

    Пока что запомним этот факт. Рассмотрим произвольные \( x_1, x_2 \in \langle A, + \infty ): \; x_1<x_2\). Зафиксируем \( x_1\). 
    \[ F\left( x_2\right)= \dfrac{ f(x_2)-f(x_1)}{ x_2-x_1} \text{ строго возрастает на } (x_1, + \infty ) \text{ по \hyperlink{thm:three_hord_crl}{следствию из Леммы о трёх хордах.}}\]

    То, что если функция возрастает везде, то она везде меньше своего предела в \( + \infty \) - не очень сложный факт. Его можно доказать прямо из определения предела 
    (или можно сослаться на теорему Вейерштрасса о пределе монотонной функции, доопределив её в \( + \infty \) чем надо, но здесь надо аккуратно). Так или иначе: 
    \[  F\left( x_2\right)= \dfrac{ f(x_2)-f(x_1)}{ x_2-x_1} \text{ строго возрастает} \implies \forall \; x_2 \in (x_1, + \infty )\quad F\left( x_2\right)< \lim\limits_{ x_2 \rightarrow + \infty } F(x_2)\]

    При \( x_2 \rightarrow + \infty\quad F(x_2) = \dfrac{ f(x_2)-f(x_1)}{ x_2-x_1} \sim \dfrac{ f(x_2)}{ x_2} \implies \lim\limits_{ x_2 \rightarrow + \infty } F(x_2)= \lim\limits_{ x_2 \rightarrow + \infty } \dfrac{ f(x_2)}{ x_2}=k\)
    по определению ассимптоты. 

    \begin{equation*}
        \begin{cases}
            \lim\limits_{ x_2 \rightarrow + \infty } F(x_2)=k\\ 
            \forall \; x_2 \in (x_1, + \infty )\quad F(x_2)< \lim\limits_{ x_2 \rightarrow + \infty } F(x_2)
        \end{cases}
        \implies 
        \forall \; x_2 \in (x_1, + \infty )\quad F(x_2)< k
    \end{equation*}

    \[ F(x_2) < k\]
    \[ \dfrac{ f(x_2)-f(x_1)}{ x_2-x_1}<k\]
    \[ \underbrace{f(x_2)-kx_2}_{g(x_2)}<\underbrace{f(x_1)-kx_1}_{g(x_1)}\]

    Мы получили, что функция \( g\) убывает на \( \langle A, + \infty )\) и её предел равен \( b\). Значит она всюду должна быть больше \( b\), иначе не выполняется определение предела (похожие соображения мы уже применяли в ходе доказательства). 
    \[ g(x)>b\]
    \[ f(x) > kx+b\quad\;  \forall \; x \in \langle A, + \infty )\]
\end{proof}
\end{document}