\documentclass[../main.tex]{subfiles}
\begin{document}
\newpage
\section{Определения и примеры несобственного интеграла с несколькими особенностями; несобственного интеграла в смысле главного значения.}
\( \Let \; - \infty \leq a <x_1 <x_2 < \ldots <x_{n-1} <b \leq + \infty,\quad a < c_1<x_1<c_2<x_2< \ldots <c_n<b,\)\\
\( f \in C\left( a,b\right) \backslash \left\{ x_1, x_2, \ldots , x_{n-1}\right\}\) 

\emph{Несобственный интеграл с несколькими особенностями} определяется как сумма интегралов с одной особенностью:
\[ \displaystyle\int\limits_{ a}^{ b} f\left( x\right)dx= \displaystyle\int\limits_{ a}^{ c_1} f\left( x\right)dx+ \displaystyle\int\limits_{ c_1}^{ x_1} f\left( x\right)dx+ \displaystyle\int\limits_{ x_1}^{ c_2} f\left( x\right)dx + \ldots + \displaystyle\int\limits_{ c_n}^{ b} f\left( x\right)dx\]

Если хотя бы один из интегралов в правой части расходится, то интеграл в левой части полагают расходящимся. 

Пример такого интеграла получить легко, можно сложить два интеграла \hyperlink{ex:converge}{из известного примера:} \( \displaystyle\int\limits_{ 0}^{ + \infty } \dfrac{ dx}{ x^p} \). 
У получившегося интеграла 2 особенности: в 0 и в \( + \infty \).

\( \Let \; c \in \left( a,b\right),\quad f \in C\left[ a,b\right] \backslash \left\{ c\right\}\). \emph{Интегралом в смысле главного значения} называется 
\[ \vp \displaystyle\int\limits_{ a}^{ b} f\left( x\right)dx = \lim\limits_{ \varepsilon \rightarrow 0} \left( \displaystyle\int\limits_{ a}^{ c- \varepsilon } f\left( x\right)dx + \displaystyle\int\limits_{ c+ \varepsilon }^{ b} f\left( x\right)dx\right)\]

Если бы это был обычный несобственный интеграл, мы бы считали его так:
\[ \displaystyle\int\limits_{ a}^{ b} f\left( x\right)dx= \displaystyle\int\limits_{ a}^{ \rightarrow c} f\left( x\right)dx+ \displaystyle\int\limits_{ \rightarrow c}^{ b} f\left( x\right)dx= \lim\limits_{ \varepsilon \rightarrow 0} \displaystyle\int\limits_{ a}^{ c- \varepsilon } f\left( x\right)dx + \lim\limits_{ \varepsilon \rightarrow 0} \displaystyle\int\limits_{ c+ \varepsilon }^{ b} f\left( x\right)dx\]

Оказывается, что сумма пределов и предел суммы это не всегда одно и то же. 

\begin{example}
    
    ~

    Допустим, мы хотим посчитать \( \displaystyle\int\limits_{ -1}^{ 1} \dfrac{ dx}{ x} \). У него есть особенность в точке 0. Если считать его как несобственный:
    \[ \displaystyle\int\limits_{ -1}^{ 1} \dfrac{ dx}{ x} = \underbrace{\displaystyle\int\limits_{ -1}^{ \rightarrow 0} \dfrac{ dx}{ x}}_{+ \infty } + \underbrace{\displaystyle\int\limits_{ \rightarrow 0}^{ 1} \dfrac{ dx}{ x}}_{- \infty }= + \infty + (- \infty ) \]

    Получается неопределённость, потому что эти две бесконечности ничего друг про друга не знают. Теперь посчитаем интеграл в смысле главного значения: 
    \[ \vp \displaystyle\int\limits_{ -1}^{ 1} \dfrac{ dx}{ x} = \lim\limits_{ \varepsilon \rightarrow 0} \left( \displaystyle\int\limits_{ -1}^{ -\varepsilon } \dfrac{ dx}{ x}  + \displaystyle\int\limits_{ \varepsilon }^{ 1} \dfrac{ dx}{ x}  \right)=0\]
\end{example}

\begin{note}
    
    ~

    Этот пример и показывает разницу между несобственным интегралом и интегралом в смысле главного значения. Тем не менее, если \( \displaystyle\int\limits_{ a}^{ b}f\left( x\right)dx \) существует как несобственный, то 
    \( \vp \displaystyle\int\limits_{ a}^{ b} f\left( x\right)dx= \displaystyle\int\limits_{ a}^{ b} f\left( x\right)dx\).
\end{note}

Если \( f \in C\left[ a,b\right] \backslash \left\{ x_1, \ldots , x_n\right\},\quad a<x_1< \ldots <x_n<b\), то \\
\( \exists \; c_0=a<x_1<c_1< \ldots <c_n<x_n<b=c_{n+1}\) и интеграл в смысле главного значения с несколькими особенностями определяется как 
\[ \vp \displaystyle\int\limits_{ a}^{ b} f\left( x\right)dx= \sum\limits_{ k=1}^{ n+1} \vp \displaystyle\int\limits_{ x_{k-1}}^{ x_k} f\left( x\right)dx\]
\end{document}