\documentclass[../main.tex]{subfiles}
\begin{document}
\newpage
\section{Теорема Барроу. Теорема Ньютона-Лейбница. Линейность интеграла.}

\begin{thm}[\hypertarget{thm:barrow}{Теорема Барроу}]

    ~

    \[ \Let \; f \in C \langle A,B \rangle ,\quad a \in \langle A,B \rangle,\quad \Phi \left( x\right)= \displaystyle\int\limits_{ a}^{ x} f\left( t\right)dt \]

    Тогда \( \Phi\) дифференцируема на \( \langle A,B \rangle \) и \( \forall \; x \in \langle A,B \rangle\quad \Phi' \left( x\right)=f\left( x\right)\)
\end{thm}

\begin{proof}
    
    ~

    \begin{equation*}
        \begin{aligned}
            \Phi'\left( x_0\right)&= \lim\limits_{ x \rightarrow x_0} \dfrac{ \Phi(x)-\Phi(x_0)}{ x-x_0} = \lim\limits_{ x \rightarrow x_0} \dfrac{ 1}{ x-x_0}  \left( \displaystyle\int\limits_{ a}^{ x} f\left( t\right)dt - \displaystyle\int\limits_{ a}^{ x_0} f\left( t\right)dt\right)=\\
             &= \lim\limits_{ x \rightarrow x_0} \dfrac{ 1}{ x-x_0} \left( \displaystyle\int\limits_{ x_0}^{ a} f\left( t\right)dt + \displaystyle\int\limits_{ a}^{ x} f\left( t\right)dt\right)= \lim\limits_{ x \rightarrow x_0} \dfrac{ 1}{ x-x_0} \displaystyle\int\limits_{ x_0}^{ x} f\left( t\right)dt \oeq
        \end{aligned}
    \end{equation*}

    По \hyperlink{thm:simple_average}{теореме о среднем} \( \exists \; c\) между \( x\) и \( x_0:\quad \displaystyle\int\limits_{ x_0}^{ x} f\left( t\right)dt=f\left( c\right)\left( x-x_0\right)\). Тогда, продолжая цепочку рассуждений:
    \[ \oeq \lim\limits_{ x \rightarrow x_0} \dfrac{ 1}{ x-x_0} f\left( c\right)\left( x-x_0\right) = \lim\limits_{ x \rightarrow x_0} f\left( c\right)\]

    При этом точка \( c\) ведь между \( x\) и \( x_0\). Поэтому по непрерывности функции \( f\) это равно просто \( f\left( x_0\right)\). Итого:
    \[ \Phi'\left( x_0\right)=f\left( x_0\right)\]
\end{proof}

\begin{crl}[\hypertarget{thm:primitive_existance}{Теорема о существовании первообразной непрерывной функции}]
    \[ f \in C \langle  a,b \rangle  \implies \exists \; \text{первообразная для } f\text{ на } \langle a,b \rangle \]
\end{crl}

\begin{proof}
    
    ~

    Все первообразные по теореме Барроу представимы в виде \( F(x)=\displaystyle\int\limits_{ a}^{ x} f\left( t\right)dt + C\)
\end{proof}

\begin{thm}[\hypertarget{thm:main_thm}{Формула Ньютона-Лейбница (основная теорема анализа)}]

    ~

    \( \Let \; f \in C \langle A,B \rangle ,\quad a,b \in \langle A,B \rangle ,\quad F\) - первообразная для \( f\) на \( \langle A,B \rangle \). 

    Тогда
    \[ \displaystyle\int\limits_{ a}^{ b} f\left( x\right)dx=F\left( b\right)-F\left( a\right)\]
\end{thm}

\begin{proof}
    
    ~

    Рассмотрим функцию \( \Phi\left( x\right)= \displaystyle\int\limits_{ a}^{ x} f\left( t\right)dt\). По теореме Барроу она является первообразной для \( f\). \hyperlink{thm:primitive_structure}{Но \( F\) тоже первообразная, поэтому} \( F\left( x\right)= \Phi \left( x\right)+C\). 
    \begin{equation*}
        \begin{aligned}
            F\left( b\right)-F\left( a\right)=\left( \Phi \left( b\right)+C\right)- \left( \Phi \left( a\right) + C\right) = \Phi \left( b\right)- \Phi\left( a\right)= \displaystyle\int\limits_{ a}^{ b} f\left( t\right)dt - \underbrace{\displaystyle\int\limits_{ a}^{ a} f\left( t\right)dt}_{0}= \displaystyle\int\limits_{ a}^{ b} f\left( t\right)dt
        \end{aligned}
    \end{equation*}
\end{proof}

Выражение \( F\left( b\right)-F\left( a\right)\) часто записывают как \( F\bigg|_a^b\) и называют \emph{двойная подстановка}. 

\begin{thm}[\hypertarget{thm:defined_integral_linear}{Линейность определённого интеграла}]
    
    ~

    \( \Let \; f,g \in C \langle A,B \rangle ,\quad a,b \in \langle A,B \rangle ,\quad c_1, c_2 \in \R \)

    Тогда
    \[ \displaystyle\int\limits_{ a}^{ b} \left( c_1f\left( x\right)+c_2g\left( x\right)\right)dx= c_1 \displaystyle\int\limits_{ a}^{ b} f\left( x\right)dx + c_2 \displaystyle\int\limits_{ a}^{ b} g \left( x\right)dx\]
\end{thm}
\begin{proof}
    Всё получается из линейности непределённого интеграла и формулы Ньютона-Лейбница. 

    Пусть \( F\) - первообразная для \( f\) на \( \langle A, B \rangle \), \( G\) - первообразная для \( g\) на \( \langle A,B \rangle \). Тогда \( c_1 F + c_2G\) - первообразная для \( c_1f+c_2g\) на \( \langle A,B \rangle \).

    По формуле Ньютона-Лейбница:
    \[ \displaystyle\int\limits_{ a}^{ b} \left( c_1f\left( x\right)+c_2g\left( x\right)\right)dx=c_1F\left( x\right)+c_2G\left( x\right)\bigg|_a^b= c_1\left( F\bigg|_a^b\right)+c_2\left( G\bigg|_a^b\right)=c_1 \displaystyle\int\limits_{ a}^{ b} f\left( x\right)dx +c_2 \displaystyle\int\limits_{ a}^{ b} g \left( x\right)dx\]
\end{proof}
\end{document}