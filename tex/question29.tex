\documentclass[../main.tex]{subfiles}
\begin{document}
\newpage
\section{Интегрирование по частям в несобственном интеграле.}

\emph{ Интегрирование по частям.} \( \Let \; u(x), v(x) \in C^1[a, b)\quad -\infty < a < b \leq +\infty.\) \;
Тогда:
\[ \displaystyle\int\limits_{ a}^{ \rightarrow b} udv = uv \bigg |^b_a \; - \; \displaystyle\int\limits_{ a}^{ \rightarrow b} vdu\]

Если два из трёх пределов существуют в конечном смысле, то существует и третий.

\vspace{5mm}

\begin{proof}

    ~

    Для доказательства достаточно устремить A к b слева в формуле выше. \\
    \[ \displaystyle\int\limits_{ a}^{ A} udv = uv \bigg |^A_a \; - \; \displaystyle\int\limits_{ a}^{ A} vdu\]
\end{proof}

\vspace{5mm}

\begin{example}

    ~

    \( \displaystyle\int\limits_{0}^{ 1} \ln(x)dx = \underbrace{x\ln(x) \bigg |^1_0}_{0} - \underbrace{\displaystyle\int\limits_{ 0}^{ 1} xd\ln(x)}_{ \int\limits_{ 0}^{ 1} 1dx } = -1\) \\
    Первый переход нужно писать сначала со знаком вопроса, так как нужно еще проверить пределы на сходимость.
\end{example}

\end{document}