\documentclass[../main.tex]{subfiles}
\begin{document}
\newpage
\section{Индуцированная топология, относительные внутренние точки, внутренности, замыкания.}
\( \Let \; \left( X, \Omega \right)\) - топологическое пространство, \( X_0 \subseteq X,\quad \Omega _0=\left\{ O\cap X_0\right\}_{O \in \Omega }\). Тогда \( \Omega _0\) называется \emph{индуцированной топологией}. Проверим, что это действительно топология:
\begin{enumerate}
    \item \( \varnothing , X_0 \in \Omega _0\)
    \item \( \underset{ \alpha \in A}{\cup}\left\{ O_{ \alpha } \cap X_0\right\}= \left( \underset{ \alpha \in A}{\cup} O_{ \alpha }\right) \cap X_0 \in \Omega _0\), потому что \( \underset{ \alpha \in A}{\cup} O_{ \alpha } \in \Omega \)
    \item \( \underset{ \alpha \in A}{\cap} \left( O_{ \alpha } \cap X_0 \right)= \left( \underset{ \alpha \in A}{\cap}O_{ \alpha }\right) \cap X_0 \in \Omega _0\), потому что \( \underset{ \alpha \in A}{\cap} O_{ \alpha } \in \Omega \)
\end{enumerate} 
\end{document}
