\documentclass[../main.tex]{subfiles}
\begin{document}
\newpage
\section{Интегрирование выражений, содержащих радикалы от дробно-линейных функций.}
Есть способ интегрировать функции вида 
\[ R\left( \;\sqrt[N_1]{ \dfrac{ \alpha t+\beta}{ \gamma t+ \delta } }, \;\sqrt[N_2]{ \dfrac{ \alpha t+\beta}{ \gamma t+ \delta } }, \ldots , \;\sqrt[N_m]{ \dfrac{ \alpha t+\beta}{ \gamma t+ \delta } }\right),\quad \alpha , \beta , \gamma , \delta \in \R,\quad \alpha \delta - \gamma \beta \neq 0,\quad N_1, \ldots , N_m \in \mathbb{N}\]

Для этого классики рекомендуют делать замену вида 
\[ x= \;\sqrt[N]{ \dfrac{ \alpha t+\beta}{ \gamma t+ \delta } }, \text{ где }  N=\lcm\left( N_1, N_2, \ldots , N_m\right)\]

\begin{example}
    
    ~

    Есть интеграл \( \displaystyle\int\limits_{ }^{ } \dfrac{ \;\sqrt[3]{x}dx}{ 1+ \;\sqrt[]{x}} \). По совету классиков сделаем в нём замену \( y= \;\sqrt[6]{x}\):
    \[ \displaystyle\int\limits_{ }^{ } \dfrac{ \;\sqrt[3]{x}dx}{ 1+ \;\sqrt[]{x}} = \displaystyle\int\limits_{ }^{ } \dfrac{ 6y^2y^5dy}{ 1+y^3}=6 \displaystyle\int\limits_{ }^{ } \dfrac{ y^7dy}{ 1+y^3}\]

    Как мы видим, после этого мы получили интеграл от рациональной функции. А его мы \hyperlink{thm:int_ratio}{научились вычислять ранее.}
\end{example}
\end{document}