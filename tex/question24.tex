\documentclass[../main.tex]{subfiles}
\begin{document}
\newpage
\section{Неравенства Гельдера и Коши-Буняковского для определенных и несобственных интегралов.}

\begin{thm}[Неравенство Гёльдера для интегралов]
    
    ~

    \( \Let \; f,g \in C\left[ a,b\right],\quad p,q > 1,\quad \dfrac{ 1}{ p} + \dfrac{ 1}{ q} =1\)

    Тогда
    \[ \left| \displaystyle\int\limits_{ a}^{ b} f\left( x\right)g \left( x\right)dx\right| \leq \left( \displaystyle\int\limits_{ a}^{ b} \left| f \left( x\right)\right|^pdx\right)^{ \frac{ 1}{ p} } \cdot \left( \displaystyle\int\limits_{ a}^{ b} \left| g \left( x\right)\right|^{ q }\right)^{ \frac{ 1}{ q} }\]
\end{thm}

\begin{proof}
    
    ~

    Доказывать будем по определению предела через суммы Римана. Пусть \( X=\left( a=x_0, x_1, \ldots ,x_n=b\right)\) - произвольное дробление отрезка \( \left[ a,b\right]\). 

    \hyperlink{thm:Gelder}{В неравенстве Гёльдера для конечных сумм} возьмём \( a_k=\left| f\left( x_k\right)\right| (\Delta x_k)^{ \frac{ 1}{ p} },\quad b_k=\left| g \left( x_k\right)\right|( \Delta x_k)^{ \frac{ 1}{ q} }\):
    \[ \left|\sum\limits_{ k=0}^{ n} f\left( x_k\right) g \left( x_k\right) ( \Delta x_k)^{ \overbrace{\frac{ 1}{ p} + \frac{ 1}{ q}}^1 }\right| \leq \left( \sum\limits_{ k=0}^{ n} \left| f\left( x_k\right)\right|^p\Delta x_k\right)^{ \frac{ 1}{ p} } \left( \sum\limits_{ k=0}^{ n} \left| g \left( x_k\right)\right|^q \Delta x_k\right)^{ \frac{ 1}{ q} }\]

    Тогда слева под модулем записана интегральная сумма Римана для функции \( f \cdot g\), справа в скобках записаны интегральные суммы функций \( \left| f\left( x\right)\right|^p\) и \( \left| g \left( x\right)\right|^q\).
    Поэтому сделая предельный переход, получим: 

    \[ \left| \displaystyle\int\limits_{ a}^{ b} f\left( x\right)g \left( x\right)dx\right| \leq \left( \displaystyle\int\limits_{ a}^{ b} \left| f \left( x\right)\right|^pdx\right)^{ \frac{ 1}{ p} } \cdot \left( \displaystyle\int\limits_{ a}^{ b} \left| g \left( x\right)\right|^{ q }\right)^{ \frac{ 1}{ q} }\]
\end{proof}

\begin{crl}[Неравенство Коши-Буняковского для интегралов]
    
    ~

    \( \Let \; f, g \in C\left[ a,b\right]\).

    Тогда
    \[ \left| \displaystyle\int\limits_{ a}^{ b} f\left( x\right) g \left( x\right)dx\right| \leq \;\sqrt[]{ \displaystyle\int\limits_{ a}^{ b} f^2\left( x\right)dx} \cdot \;\sqrt[]{ \displaystyle\int\limits_{ a}^{ b} g^2\left( x\right)dx}\]
\end{crl}
\end{document}