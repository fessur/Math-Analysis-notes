\documentclass[../main.tex]{subfiles}
\begin{document}
\newpage
\section{Характеристика сходимости положительного ряда через ограниченность последовательности частичных сумм. Признак сравнения и его следствия и частные случаи.}

\begin{thm}[Критерий сходимости положительного ряда]
    
    ~

    \( \Let \; \sum\limits_{ k=1}^{ \infty } a_k\) - ряд, \( \forall \; k \in \N\quad a_k \geq 0,\quad S_n = \sum\limits_{ k=1}^{ n} a_k\).

    Тогда
    \[ \sum\limits_{ k=1}^{ \infty } a_k \text{ сходится } \Longleftrightarrow \left\{ S_n\right\} \text{ ограничена сверху}\]
\end{thm}
\begin{proof}
    
    ~
    \\\( \boxed{\Longrightarrow}\)

    Сходящаяся последовательность ограничена, а так как \( \forall \; k\quad a_k \geq 0 \Longrightarrow S_k \geq 0\), достаточно ограниченности сверху.
    \\ \( \boxed{\Longleftarrow}\)

    Последовательность \( \left\{ S_n\right\}\) ограничена и возрастает. Значит она сходится к своему супремуму.
\end{proof}

\begin{note}
    Если \( \forall \; k\quad a_k \geq 0\), то \( \sum\limits_{ k=1}^{ \infty } a_k\) сходится, либо равен \( + \infty \).
\end{note}

\begin{thm}[Признак сравнения]
    
    ~

    \( \Let \; \forall \; k \in \N\quad a_k \geq 0, \;b_k \geq 0,\quad a_k\underset{k \rightarrow \infty }{=}O\left( b_k\right)\).

    Тогда

    \begin{equation*}
        \begin{aligned}
            &\sum\limits_{ k=1}^{ \infty } b_k \text{ сходится } \Longrightarrow \sum\limits_{ k=1}^{ \infty } a_k \text{ сходится}\\
            & \sum\limits_{ k=1}^{ \infty} a_k \text{ расходится } \Longrightarrow \sum\limits_{ k=1}^{ \infty } b_k \text{ расходится}
        \end{aligned}
    \end{equation*}
\end{thm}
\begin{proof}
    
    ~

    По определению О-большого \( \exists \; C, N:\quad \forall \; k \geq N\quad a_k \leq C\cdot b_k\). 

    Пусть \( S_n = \sum\limits_{ k=N}^{ n} a_k,\quad T_n= \sum\limits_{ k=N}^{ n} b_k \Longrightarrow S_n \leq C\cdot T_n\).

    \( \sum\limits_{ k=1}^{ \infty } b_k\) сходится \( \Longrightarrow \sum\limits_{ k=N}^{ \infty } b_k\) сходится \( \Longrightarrow \exists \; \lim\limits_{ n \rightarrow \infty } T_n= \sup\limits_{ } T_n =T\in \R \Longrightarrow S_n \leq C\cdot T\)

    То есть последовательность \( \left\{ S_n\right\}\) ограничена и возрастает \( \Longrightarrow \sum\limits_{ k=N}^{ \infty } a_k\) сходится \( \Longrightarrow \sum\limits_{ k=1}^{ \infty } a_k\) сходится. 

    Если же \( \sum\limits_{ k=1}^{ \infty } a_k\) расходится, то \( \lim\limits_{ n \rightarrow \infty } S_n = + \infty \). После предельного перехода получим \( + \infty \leq C\cdot T\), значит \( T = + \infty \), то есть \( \sum\limits_{ k=N}^{ \infty} b_k\) расходится \( \Longrightarrow \sum\limits_{ k=1}^{ \infty } b_k\) расходится. 
\end{proof}

\begin{thm}[Признак сравнения в ассимптотической форме]
    
    ~

    \( \Let \; a_k \geq 0,\quad \forall \; k\quad b_k \underset{k \rightarrow \infty }{\sim}a_k\).

    Тогда 
    \[ \sum\limits_{ k=1}^{ \infty } a_k\text{ сходится } \Longleftrightarrow \sum\limits_{ k=1}^{ \infty } b_k\text{ сходится }\]
\end{thm}
\begin{proof}
    
    ~

    \( b_k \sim a_k \Longrightarrow b_k = \varphi _k \cdot a_k,\quad \varphi \longrightarrow 1\Longrightarrow b_k \geq 0\) начиная с некоторого номера. Тогда всё следует из предыдущего признака сравнения, т.к. \( b_k \sim a_k \Longleftrightarrow a_k=O\left( b_k\right),\quad b_k=O\left( a_k\right)\).
\end{proof}
\end{document}
