\documentclass[../main.tex]{subfiles}
\begin{document}
\newpage
\section{Интегрирование по частям и замена переменной в определенном интеграле, примеры.}
\begin{thm}[\hypertarget{thm:def_by_parts}{Интегрирование по частям в определённом интеграле}]
    
    ~

    \( \Let \; u\left( x\right), v\left( x\right) \in C^1\left[ a,b\right]\)

    Тогда
    \[ \displaystyle\int\limits_{ a}^{ b} u\left( x\right)v'\left( x\right)dx=u\left( b\right)v\left( b\right)-u\left( a\right)v\left( a\right)- \displaystyle\int\limits_{ a}^{ b} v\left( x\right)u'\left( x\right)dx\]

    Или по простому 
    \[ \displaystyle\int\limits_{ a}^{ b} udv=uv\bigg|_a^b- \displaystyle\int\limits_{ a}^{ b} vdu\]
\end{thm}

\begin{proof}
    
    ~

    \( \left( uv\right)'=u'v+uv'\), все функции непрерывны. \hypertarget{thm:defined_integral_linear}{Тогда по линейности определённого интеграла}
    \begin{equation}\label{lab:eq:def_int_by_parts}
        \displaystyle\int\limits_{ a}^{ b} \left( uv\right)'\left( x\right)dx= \displaystyle\int\limits_{ a}^{ b} \left( u'\left( x\right)v\left( x\right)+u\left( x\right)v'\left( x\right)\right)dx= \displaystyle\int\limits_{ a}^{ b} u'\left( x\right)v\left( x\right)dx+ \displaystyle\int\limits_{ a}^{ b} u\left( x\right)v'\left( x\right)dx
    \end{equation}

    \( uv\) - первообразная для \( \left( uv\right)'\), поэтому по \hyperlink{thm:main_thm}{формуле Ньютона-Лейбница} \\
    \( \displaystyle\int\limits_{ a}^{ b} \left( uv\right)'\left( x\right)dx=uv\bigg|_a^b\). Совмещая это с \ref{lab:eq:def_int_by_parts}, получаем 
    \[ uv\bigg|_a^b= \displaystyle\int\limits_{ a}^{ b}u'\left( x\right) v\left( x\right)dx+ \displaystyle\int\limits_{ a}^{ b} u\left( x\right)v'\left( x\right)dx\]
    \[ \displaystyle\int\limits_{ a}^{ b} u\left( x\right)v'\left( x\right)dx=uv\bigg|_a^b- \displaystyle\int\limits_{ a}^{ b} u'\left( x\right)v\left( x\right)dx\]
\end{proof}

\begin{example}
    
    ~

    Требуется вычислить интеграл \( \displaystyle\int\limits_{ 0}^{ x_0} \cos \left( mx\right)e^{ \lambda x}dx,\quad m, \lambda \in \R \;\backslash \left\{ 0\right\}\). Обозначим этот интеграл \( I\). 
    Чтобы его посчитать, ничего особенного применять не надо. Нужна только выдержка и формула интегрирования по частям. 
    \begin{equation*}
        \begin{aligned}
            I&= \displaystyle\int\limits_{ 0}^{ x_0} \cos \left( mx\right) d\left( \dfrac{ e^{\lambda x}}{ \lambda } \right)= \dfrac{ 1}{ \lambda } \left( \cos \left( mx\right)e^{ \lambda x}\bigg|_0^{x_0}- \displaystyle\int\limits_{ 0}^{ x_0} e^{ \lambda x} d\left( \cos \left( mx\right)\right)\right)= \\
            &=\dfrac{ 1}{ \lambda } \left( \underbrace{\cos \left( mx_0\right)e^{ \lambda x_0}-1}_{g\left( x_0\right)}+ m \displaystyle\int\limits_{ 0}^{ x_0} \sin\left( mx\right)d\left( \dfrac{ e^{\lambda x}}{ \lambda } \right)\right) =\\
            &= \dfrac{ 1}{ \lambda } \left( g \left( x_0\right)+ \dfrac{ m}{ \lambda } \left( \sin\left( mx\right)e^{ \lambda x}\bigg|_0^{x_0}- \displaystyle\int\limits_{ 0}^{ x_0} e^{ \lambda x} d\left( \sin\left( mx\right)\right)\right)\right)=\\
            &= \dfrac{ g(x_0)}{ \lambda } + \dfrac{ m}{ \lambda ^2}\underbrace{ \sin\left( mx_0\right)e^{ \lambda x_0}}_{h\left( x_0\right)}- \dfrac{ m^2}{ \lambda ^2} \underbrace{\displaystyle\int\limits_{ 0}^{ x_0} e^{ \lambda x} \cos \left( mx\right)dx}_I =\\
            &= \dfrac{ g(x_0)}{ \lambda } + \dfrac{ m}{ \lambda ^2} h\left( x_0\right) + \dfrac{ m^2}{ \lambda ^2} I 
        \end{aligned}
    \end{equation*}

    Итого
    \[ I= \dfrac{ 1}{ 1- \dfrac{ m^2}{ \lambda ^2} } \left( \dfrac{ g\left(x_0\right)}{ \lambda } + \dfrac{ m}{ \lambda ^2} h\left( x_0\right)\right)\]
\end{example}

\begin{thm}[Замена переменной в определённом интеграле]
    
    ~

    \( \Let \; X, T\) - промежутки\(,\quad f \in C\left( X\right),\quad \varphi : T \longrightarrow X,\quad  \varphi \in C^1\left( T\right),\quad \alpha , \beta \in T\)

    Тогда
    \[ \displaystyle\int\limits_{ \varphi \left( \alpha\right)}^{ \varphi \left( \beta \right)} f\left( x\right)dx= \displaystyle\int\limits_{ \alpha }^{ \beta } f\left( \varphi \left( t\right)\right) \varphi '\left( t\right)dt\]
\end{thm}

\begin{proof}
    
    ~

    \hyperlink{thm:primitive_existance}{\( f \in C\left( X\right) \implies \) у \( f\) существует первообразная F на \( X\).}

    \hyperlink{thm:main_thm}{По формуле Ньютона-Лейбница:}
    \[ \displaystyle\int\limits_{ \varphi \left( \alpha \right)}^{ \varphi \left( \beta \right)} f\left( x\right)dx=F\bigg|_{ \varphi \left( \alpha \right)}^{ \varphi \left( \beta \right)}=F\left( \varphi \left( \beta \right)\right)-F\left( \varphi \left( \alpha \right)\right)\]

    \hyperlink{thm:undef_int_change}{По теореме о замене переменной в неопределённом интеграле \( F \circ \varphi \) - первообразная} для \( f\left( \varphi \left( t\right)\right) \varphi '\left( t\right)\). \hyperlink{thm:main_thm}{Тогда по формуле Ньютона-Лейбница:}
    \[ \displaystyle\int\limits_{ \alpha }^{ \beta } f\left( \varphi \left( t\right)\right) \varphi '\left( t\right)dt=F \circ \varphi \bigg|_{ \alpha }^{ \beta }=F\left( \varphi \left( \beta \right)\right)-F\left( \varphi \left( \alpha \right)\right)\]

    \begin{equation*}
        \begin{aligned}
            &\displaystyle\int\limits_{ \varphi \left( \alpha \right)}^{ \varphi \left( \beta \right)} f\left( x\right)dx=F\bigg|_{ \varphi \left( \alpha \right)}^{ \varphi \left( \beta \right)}=F\left( \varphi \left( \beta \right)\right)-F\left( \varphi \left( \alpha \right)\right)\\
            &\displaystyle\int\limits_{ \alpha }^{ \beta } f\left( \varphi \left( t\right)\right) \varphi '\left( t\right)dt=F \circ \varphi \bigg|_{ \alpha }^{ \beta }=F\left( \varphi \left( \beta \right)\right)-F\left( \varphi \left( \alpha \right)\right)
        \end{aligned}
        \implies 
        \displaystyle\int\limits_{ \varphi \left( \alpha \right)}^{ \varphi \left( \beta \right)} f\left( x\right)dx =\displaystyle\int\limits_{ \alpha }^{ \beta } f\left( \varphi \left( t\right)\right) \varphi '\left( t\right)dt
    \end{equation*}
\end{proof}

\begin{crl}
    
    ~

    Если в условии предыдущей теоремы \( \varphi \) - биекция, то 
    \[ \displaystyle\int\limits_{ a}^{ b} f\left( x\right)dx= \displaystyle\int\limits_{ \varphi ^{-1}\left( \alpha \right)}^{ \varphi ^{-1}\left( \beta \right)} f\left( \varphi \left( t\right)\right) \varphi '\left( t\right)dt\]
\end{crl}

\begin{example}
    \[ \displaystyle\int\limits_{ 0}^{ 2} \dfrac{ x^2dx}{ x^3+1} \underset{\;t=x^3+1\;}{=} \dfrac{ 1}{ 3} \displaystyle\int\limits_{ 1}^{ 9} \dfrac{ dt}{ t} = \dfrac{ 1}{ 3} \ln \left| t\right| \;\bigg|_1^9= \dfrac{ 2}{ 3} \ln 3\]
\end{example}

\begin{example}
    
    ~

    Требуется вычислить \( \displaystyle\int\limits_{ 0}^{ 1} \dfrac{ dx}{ \left( 1+x^2\right)^{ \frac{ 3}{ 2} }}\). 

    Так как \( \sh t\) - строго возрастающая функция на \( \R \), существует обратная к ней функция. Поэтому можем заменить \( x\) на \( \sh t\) в интеграле.  
    \begin{equation*}
        \begin{aligned}
            \displaystyle\int\limits_{ 0}^{ 1} \dfrac{ dx}{ \left( 1+x^2\right)^{ \frac{ 3}{ 2} }}&= \displaystyle\int\limits_{ 0}^{ \sh^{-1}(1)} \dfrac{ \ch t}{ \left( 1+\sh^2 t\right)^{ \frac{ 3}{ 2} }} dt=\displaystyle\int\limits_{ 0}^{ \sh^{-1}(1)} \dfrac{ \ch t}{ \ch^3 t} dt= \displaystyle\int\limits_{ 0}^{ \sh^{-1}(1)} \dfrac{ dt}{ \ch^2 t} = \\
            &=\th t \bigg|_0^{\sh^{-1}(1)}= \dfrac{ \sh(\sh^{-1}(1))}{ \ch(\sh^{-1}(1))} =\dfrac{ \sh(\sh^{-1}(1))}{ \;\sqrt[]{1+ \sh^2\left( \sh^{-1}(1)\right)}} = \dfrac{ 1}{ \;\sqrt[]{2}} 
        \end{aligned}
    \end{equation*}
\end{example}
\end{document}