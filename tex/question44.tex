\documentclass[../main.tex]{subfiles}
\begin{document}
\newpage
\section{Работа силы.}
В этом билете будет рассматриваться ситуация, когда вектор силы параллелен одной и той же прямой (то есть вектор \( F\) можно отождествить с числом). 

В случае, если \( F\) постоянна и равна \( c\), работа силы \( F\) по переносу материальной точки из точки \( a\) в точку \( b\) равна \( c \cdot \left( b-a\right)\).

Если же сила \( F\) переменная, то \( \Phi\left( \left[ a,b\right]\right)\) - работа силы по переносу частицы из точки \( a\) в точку \( b\) - аддитивная функция промежутка. При этом 
\[ \min\limits_{ \left[ a,b\right]} F \cdot \left( b-a\right) \leq \Phi\left( \left[ a,b\right]\right) \leq \max\limits_{ \left[ a,b\right]} F \cdot \left( b-a\right)\]

\hyperlink{thm:density}{Тогда по признаку плотности} \( F\) - плотность для \( \Phi\) и значит 
\[ \boxed{\Phi\left( \left[ a,b\right]\right)= \displaystyle\int\limits_{ a}^{ b} F\left( x\right)dx}\]
\end{document}