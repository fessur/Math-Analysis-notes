\documentclass[../main.tex]{subfiles}
\begin{document}
\newpage
\section{Первая теорема о среднем.}

\begin{thm}[\hypertarget{thm:first_average}{Первая теорема о среднем}]
    
    ~

    \( \Let \; f,g \in C\left[ a,b\right],\quad g \left( x\right) \geq 0\) и \( \exists \; x_0 \in \left[ a,b\right]:\quad g \left( x_0\right) >0\)

    Тогда 
    \[ \exists \; c \in \left[ a,b\right]:\quad \displaystyle\int\limits_{ a}^{ b} f\left( x\right)g \left( x\right)dx = f\left( c\right) \cdot \displaystyle\int\limits_{ a}^{ b} g \left( x\right)dx\]
\end{thm}

\begin{proof}
    
    ~

    Обозначим \( M = \max\limits_{ \left[ a,b\right]} f,\quad m= \min\limits_{ \left[ a,b\right]} f\). Оба достигаются по теореме Вейерштрасса, т.к. \( f \in C\left[ a,b\right]\)

    \( \forall \; x \in \left[ a,b\right]\quad m \leq f\left( x\right) \leq M,\quad g \left( x\right) \geq 0 \implies m g \left( x\right) \leq f\left( x\right)g \left( x\right) \leq M g \left( x\right)\)

    Тогда из монотонности интеграла:
    \[ m \cdot \displaystyle\int\limits_{ a}^{ b} g \left( x\right)dx = \displaystyle\int\limits_{ a}^{ b} m g \left( x\right)dx \leq \displaystyle\int\limits_{ a}^{ b} f\left( x\right)g \left( x\right)dx \leq \displaystyle\int\limits_{ a}^{ b} Mg \left( x\right)dx = M \cdot \displaystyle\int\limits_{ a}^{ b} g \left( x\right)dx\]

    Обозначим \( I = \displaystyle\int\limits_{ a}^{ b} f\left( x\right)g \left( x\right)dx,\quad J= \displaystyle\int\limits_{ a}^{ b} g \left( x\right)dx\). \hyperlink{thm:def_int_prop}{По 5 свойству интеграла} \( J > 0\).

    \( \dfrac{ I}{ J} \in \left[ m, M\right]\). При этом \( \dfrac{ I}{ J} \) дифференцируема, а значит непрерывна, а значит по теореме о промежуточном значении \( \exists \; t \in \left[ m,M\right]:\quad \dfrac{ I}{ J} =t\). Функция \( f\) непрерывна, значит по той же теореме о промежуточном значении должна принимать все значения между \( m\) и \( M\). Значит \( \exists \; c \in \left[ a,b\right]: \dfrac{ I}{ J} =f\left( c\right)\)
    \[ I=f \left( c\right) \cdot J\]
    \[ \displaystyle\int\limits_{ a}^{ b} f\left( x\right)g \left( x\right)dx = f\left( c\right) \cdot  \displaystyle\int\limits_{ a}^{b } g \left( x\right)dx\]
\end{proof}
\end{document}