\documentclass[../main.tex]{subfiles}
\begin{document}
\newpage
\section{Сходимость несобственных интегралов от степенной функции по промежуткам \((0; 1)\) и \( (1; + \infty )\)}
    
\begin{enumerate}
    \item \( \displaystyle\int\limits_{ 1}^{ +\infty} \dfrac{ dx}{ x^p}, p \in \mathbb{R}.\)
    
    ~

    \( \displaystyle\int\limits_{ 1}^{ +\infty} \dfrac{ dx}{ x^p} = \lim\limits_{ c \; \rightarrow \; +\infty} \displaystyle\int\limits_{ 1}^{ c} \dfrac{ dx}{ x^p} = \lim\limits_{ c \; \rightarrow \; +\infty} 
    \displaystyle\begin{cases}
        \dfrac{x^{-p + 1} }{ -p + 1} \bigg|_1^c, & \; \text{если } p \neq 1; \\
        \ln(x) \bigg|_1^c, & \; \text{если } p = 1.
    \end{cases}
    \quad =\quad 
    \begin{cases}
    \dfrac{ 1}{ -1 + p}, & \; \text{если } p > 1; \\
    +\infty, & \text{если } p < 1; \\
    +\infty, & \text{если } p = 1. 
    \end{cases}
    \)

    ~
    
    \boxed{\displaystyle\int\limits_{ 1}^{ +\infty} \dfrac{ dx}{ x^p}\quad \text{сходится} \Longleftrightarrow p > 1.}
    
    ~

    \item \( \displaystyle\int\limits_{ \rightarrow 0}^{ 1} \dfrac{ dx}{ x^p}, p \in \mathbb{R}. \)
    
    ~

    \( \displaystyle\int\limits_{ \rightarrow 0}^{ 1} \dfrac{ dx}{ x^p} = \lim\limits_{ c \; \rightarrow \; 0+} \displaystyle\int\limits_{ c}^{ 1} \dfrac{ dx}{ x^p} = \lim\limits_{ c \; \rightarrow \; 0+} 
    \begin{cases}
        \dfrac{ x^{-p + 1}}{ -(p - 1)} \bigg|_c^1, & \; \text{если } p \neq 1; \\
        \ln(x) \bigg|_c^1, & \; \text{если } p = 1.
    \end{cases}
    \quad =\quad
    \begin{cases}
        \dfrac{ 1}{ 1 - p}, & \; \text{если } p < 1; \\
        +\infty, & \; \text{если } p > 1; \\
        +\infty, & \; \text{если } p = 1.
    \end{cases}
    \)

    ~

    \boxed{\displaystyle\int\limits_{ \rightarrow 0}^{ 1} \dfrac{ dx}{ x^p}\quad \text{сходится} \Longleftrightarrow p < 1.}
\end{enumerate}


\end{document}