\documentclass[../main.tex]{subfiles}
\begin{document}
\newpage
\section{Признак Лейбница. Нахождение суммы ряда \( \sum_{ k=1}^{ \infty } \frac{ (-1)^{k-1}}{ k} \) и некоторых его перестановок}

\begin{thm}[Признак Лейбница]

    ~
    
    \( \Let \; b_k \longrightarrow 0,\quad \forall \; k\quad b_k \geq 0,\quad b_k\) монотонна. 

    Тогда 
    \begin{enumerate}
        \item \( \sum\limits_{ k=1}^{ \infty } \left( -1\right)^{k-1}b_k\) сходится.
        \item \( \forall \; N \in \N\quad \left| \sum\limits_{ k=N}^{ \infty } \left( -1\right)^{k-1}b_k\right| \leq b_N\), а знак остатка \( \sum\limits_{ k=N}^{ \infty } \left( -1\right)^{k-1}b_k\) совпадает со знаком первого слагаемого, то есть \( \left( -1\right)^{N-1}b_N\).
    \end{enumerate}
\end{thm}
\begin{proof}
    
    ~

    \begin{enumerate}
        \item Это верно \hyperlink{thm:series_dirihle}{по признаку Дирихле}: частичные суммы \( \sum\limits_{ k=1}^{ \infty } \left( -1\right)^{k-1}\) ограничены, 
        \par \( b_k \longrightarrow 0,\quad b_k\) монотонна. 
        \item Не умаляя общности можно считать \( N = 1\), потому что для других \( N\) можно рассмотреть ряд, являющийся остатком ряда \( \sum\limits_{ k=1}^{ \infty } \left( -1\right)^{k-1}b_k\), начинающийся с номера \( N\). Для этого остатка член с номером \( N\) уже будет первым членом ряда, а остальные условия будут по прежнему выполнены (разве что придётся вынести минус, чтобы \( b_k\) по прежнему было неотрицательным). Так как последовательность \( b_k\) монотонна, неотрицательна и стремится к 0, она может только убывать. По \hyperlink{thm:series_group}{теореме о группировке членов ряда} скобки можно ставить как угодно. Тогда:
        \begin{equation*}
            \begin{aligned}
                \underbrace{b_1-b_2}_{ \geq 0}+\underbrace{b_3-b_4}_{ \geq 0}+\underbrace{b_5-b_6}_{ \geq 0}+ \ldots\quad &\geq 0 \\ 
                b_1 - \underbrace{\left( b_2-b_3\right)}_{ \geq 0}-\underbrace{\left( b_4-b_5\right)}_{ \geq 0}- \ldots\quad  &\leq b1
            \end{aligned}
        \end{equation*}

        Таким образом, \( \sum\limits_{ k=1}^{ \infty } \left( -1\right)^{k-1}b_k \;\in [0, b_1]\).
    \end{enumerate}
\end{proof}

Если \( \varphi : \N \longrightarrow \N\) - биекция, то ряд \( \sum\limits_{ k=1}^{ \infty } a_{ \varphi \left( k\right)}\) называется \emph{перестановкой} ряда \( \sum\limits_{ k=1}^{ \infty } a_k\). 

\begin{example}[Нахождение суммы ряда и его перестановки]
    
    ~

    Рассмотрим ряд \( \sum\limits_{ k=1}^{ \infty } \dfrac{ (-1)^{k-1}}{ k} \). Он сходится \hyperlink{thm:series_dirihle}{по признаку Дирихле}. \hyperlink{thm:series_group}{Значит все его группировки тоже сходятся.} Обозначим \( S_n= \sum\limits_{ k=1}^{ n} \dfrac{ \left(-1\right)^{k-1}}{ k} \). Рассмотрим группировку "по двое" и посчитаем сумму такого ряда:

    \[ S_{2n}=1 - \dfrac{ 1}{ 2} + \dfrac{ 1}{ 3} - \dfrac{ 1}{ 4} + \dfrac{ 1}{ 5} - \ldots + \dfrac{ 1}{ 2n-1} - \dfrac{ 1}{ 2n}= \underbrace{\sum\limits_{ k=1}^{ 2n} \dfrac{ 1}{ k}}_{H_{2n}} - \underbrace{2\sum\limits_{ k=1}^{ n} \dfrac{ 1}{ 2k}}_{H_n}\]

    \[ S_{2n}=H_{2n}-H_n\]

    \hyperlink{ex:harmonic_sim}{Как мы знаем} \( H_n = \underbrace{\displaystyle\int\limits_{ 1}^{ n+1} \dfrac{ dx}{ x} }_{ \ln \left( n+1\right)}+C_{\textrm{Э}}+o\left( 1\right) \). Тогда:

    \[ S_{2n}= \ln \left( 2n+1\right)+C_{\textrm{Э}}+o\left( 1\right)- \ln \left( n+1\right)-C_{\textrm{Э}}+o\left( 1\right)= \ln \left( \dfrac{ 2+\frac{1}{n}}{ 1+ \frac{ 1}{ n} }\right) + o\left( 1\right) \longrightarrow \ln 2\]

    \[ \boxed{ \sum\limits_{ k=1}^{ \infty } \dfrac{ \left(-1\right)^{k-1}}{ k} = \ln 2}\]

    Теперь посмотрим на такую перестановку этого ряда: \( 1 - \dfrac{ 1}{ 2} - \dfrac{ 1}{ 4} + \dfrac{ 1}{ 3} - \dfrac{ 1}{ 6} - \dfrac{ 1}{ 8} + \dfrac{ 1}{ 5} - \dfrac{ 1}{ 10} - \dfrac{ 1}{ 12} + \ldots \) Рассмотрим группировку полученного ряда "по три":
    \begin{equation*}
        \begin{aligned}
            & S_{3n}= \left( 1 - \dfrac{ 1}{ 2} - \dfrac{ 1}{ 4} \right)+ \left(\dfrac{ 1}{ 3} - \dfrac{ 1}{ 6} - \dfrac{ 1}{ 8}\right) + \left(\dfrac{ 1}{ 5} - \dfrac{ 1}{ 10} - \dfrac{ 1}{ 12}\right) + \ldots = \\ 
            & = \underbrace{\sum\limits_{ k=1}^{ n} \dfrac{ 1}{ 2k-1}}_{x_n} - \underbrace{\sum\limits_{ k=1}^{ n} \dfrac{ 1}{ 4k-2}}_{ \frac{ 1}{ 2}x_n } - \underbrace{\sum\limits_{ k=1}^{ n} \dfrac{ 1}{ 4k} }_{ \frac{ 1}{ 4}H_n}= \dfrac{ 1}{ 2} x_n- \dfrac{ 1}{ 4} H_n
        \end{aligned}
    \end{equation*}

    Заметим, что \( \sum\limits_{ k=1}^{ n} \dfrac{ 1}{ 2k-1} + \underbrace{\sum\limits_{ k=1}^{ n} \dfrac{ 1}{ 2k}}_{ \frac{ 1}{ 2}H_n } = \sum\limits_{ k=1}^{ 2n} \dfrac{ 1}{ k} = H_{2n} \Longrightarrow x_n= H_{2n}- \dfrac{ 1}{ 2} H_n\). Тогда 

    \[ S_{3n}=\dfrac{ 1}{ 2} x_n- \dfrac{ 1}{ 4} H_n= \dfrac{ 1}{ 2} H_{2n}- \dfrac{ 1}{ 2} H_n = \dfrac{ 1}{ 2} \left( H_{2n}-H_n\right) \longrightarrow \dfrac{ \ln2}{ 2} \]

    Можно рассмотреть и другие подподследовательности последовательности \( \left\{ S_n\right\}\):

    \[ S_{3n+1}= S_{3n}+ \dfrac{ 1}{ 2n+1} \longrightarrow \dfrac{ \ln2}{ 2},\quad S_{3n+2}=S_{3n+1}+ \dfrac{ 1}{ 4n+2} \longrightarrow \dfrac{ \ln2}{ 2} \]

    Подпоследовательности \( \left\{ S_{3n}\right\},\quad \left\{ S_{3n+1}\right\},\quad \left\{ S_{3n+2}\right\}\) вместе дают последовательность \( \left\{ S_n\right\} \), и все они сходятся к \( \frac{ \ln2}{ 2} \). Значит \( S_n \longrightarrow \frac{ \ln2}{ 2} \), то есть сумма ряда, полученного перестановкой, отличается от суммы исходного ряда. 
\end{example}
\end{document}
