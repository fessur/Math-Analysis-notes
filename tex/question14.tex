\documentclass[../main.tex]{subfiles}
\begin{document}
\newpage
\section{Интегрирование квадратичных иррациональностей. Подстановки Эйлера.}
Ещё один вид функций, которые мы умеем интегрировать, это квадратичные иррациональности, то есть функции, которые имеют вид \( R\left( x, \;\sqrt[]{ax^2+bx+c}\right)\). 

Для этого применяются \emph{Подстановки Эйлера}. 
\begin{enumerate}
    \item Если \( a>0\). 
    \par Эйлер рекомендует делать замену \( \;\sqrt[]{ax^2+bx+c}=\pm \;\sqrt[]{a}x \pm t\), в качестве новой переменной берётся \( t\). При этом знаки \( \pm\) не согласованы между собой 
    и выбираются как угодно, всё равно будет работать. 
    \item Если \( c>0\).
    \par Рекомендуется брать новую переменную \( t\), которая связана с \( x\) равенством \( \;\sqrt[]{ax^2+bx+c}=\pm \;\sqrt[]{c}\pm tx\). Знаки снова не согласованы. 
    \item Если \( ax^2+bx+c\) имеет вещественные корни \( x_1, x_2\). 
    \par В этой ситуации рекомендуется брать новую переменную \( t = \;\sqrt[]{a \dfrac{ x-x_1}{ x-x_2} }\).
\end{enumerate}
\end{document}